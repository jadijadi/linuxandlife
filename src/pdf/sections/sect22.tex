\section{کمک گرفتن و ادامه راه}
وجود هزاران نفر مشتاق کمک به شما، یکی از مزایای اصلی نرم‌افزارهای آزاد نسبت به نرم‌افزارهای تجاری و انحصاری است. البته در نرم‌افزارهای تجاری - مثلا ویندوز - هم شما معمولا به پشتیبانی دسترسی دارید اما این پشتیبانی در اکثر موارد ناکارا است. به دو دلیل:

کسی که سعی می‌کند از طریق تلفن پشتیبانی شما را بر عهده بگیرد در اکثر موارد از شما سواد کمتری دارد، انگلیسی بدتری حرف می‌زند و تنها سعی می‌کند از طریق پیگیری یکسری الگوریتم و نمودار (شبیه چیزی که موقع پیگیری 
\lr{Troubleshooting}
 خود ویندوز می‌بینید) مشکلات شما را حل کند. اول از شما می‌پرسد که آیا کابل پرینتر وصل است. بعد می‌پرسد که آیا درایور نصب شده و بعد از شما می‌خواهد چک کنید که کاغذ در پرینتر گیر نکرده باشد. بعد ویندوز را ری‌استارت می‌کند و در صورت حل نشدن مشکل از شما خواهد خواست که کامپیوتر را به پیش یک تعمیرکار ببرید.

سیستم‌های بسته‌ای مانند ویندوز، چندان قابل «عیب‌یابی» نیستند. شما با یک صفحه آبی روبرو می شوید و بعد باید کامپیوتر را خاموش و روشن کنید. یک پیام می‌آید که به شما می‌گوید «مشکل جدی. لطفا به مایکروسافت اطلاع دهید» (و شما هم 
\lr{Dont Send Error}
را فشار می‌دهید) و ...

اما لینوکس بر خلاف ویندوز برای عیب یابی طراحی شده. در صورت بروز هر مشکل احتمالا چندین و چند 
\lr{log}
مختلف به نمایش درمی‌آید و اشکال هم قابل تکرار کردن خواهد بود و شما می‌توانید قدم به قدم مشکل را حل کنید. اینکه کجا متوقف شوید فقط بستگی به سواد، علاقه شما به یادگیری و حوصله‌ای که به کار می‌برید دارد.
حالا برویم سراغ اینکه از چه جاهایی می‌توانید کمک بگیرید:
\subsection*{جستجو در اینترنت}
این تقریبا مهمترین جا برای یافتن جواب سوالات است. می‌خواهید بدانید چطور می‌توانید آی.پی. استاتیک به کارت شبکه بدهید؟ 
\lr{how to configure static ip}
را گوگل کنید. می‌خواهید کشف کنید که چطور باید اوبونتوی ۹.۰۴ خود را به ۹.۱۰ آپگرید کنید؟ دنبال 
\lr{upgrade 9.04 to 9.10}
 بگردید. از صفحه لاگین خود خسته شده اید و می‌خواهید تصویرش را تغییر دهید؟ «\lr{change login screen linux}» (: ساده نیست؟ کاملا ساده و سر راست.

در هنگام جستجو از کلمات کلیدی دقیق استفاده کنید. اگر دقیقا سوالتان مربوط به عملکرد یک توزیع خاص (مثلا سابایون یا مندرویا) است، آن را به عبارت مورد جستجو اضافه کنید و در غیر اینصورت، به شکل عمومی از 
\lr{linux}
استفاده کنید. از انگلیسی نوشتن و خواندن نترسید و قدم به قدم جلو بروید. سعی کنید بفهمید راه حل ارائه شده از چه طریقی کار می‌کند و حداقل به چند نتیجه جستجو توجه کنید و یکضرب امیدوار یا ناامید نشوید.
\subsection*{رفتن به انجمن‌ها}
فروم‌ها قدم دوم هستند. در فروم‌ها ده‌ها، صدها یا هزاران نفر هستند که به دلایل مختلف (از پیشبرد جامعه و اعتقاد به ثواب در آن دنیا تا اظهار فضل و کسب شهرت و حتی احساس لذت از گیک بودن) حاضرند به شما کمک کنند. معمولا از نظر فرهنگی بهتر است در فرومی که سوال می‌کنید، یک شناسه تعریف کنید و بسیار دقیق و مودب سوال خود را مطرح کنید. البته قبل از پرسیدن هر چیزی باید خوب جستجو کنید تا ببینید آیا همین سوال یا سوالی خیلی نزدیک به آن قبلا پرسیده شده یا نه.

کسانی که به شما کمک می‌کنند، داوطلب هستند پس سعی کنید شما هم حداقل با پرسیدن سوال دقیق و «هوشمندانه»، به آن‌ها کمک کنید. در عین حال مطمئنا در هر سطحی که باشید، کسانی هستند که نیازمند کمک شما هستند. بسیار خوب است که خودتان هم به نوبه خود به آن‌ها کمک کنید.

در ایران فروم‌های خوبی وجود دارند. حرفه‌ای‌ترین و قدیمی ترین انجمن، انجمن تکنوتاکس است. انجمن‌ اوبونتوکاران ایران هم این روزها یکی از فعالترین و پرکابرترین فروم‌ها است و بقیه توزیع‌ها هم فروم‌های خود را دارند که معمولا می‌توانید به سادگی با یک جستجو آن‌ها را پیدا کنید.

در سطح جهان هم تعداد خیلی زیاد فروم وجود دارد که یک جستجوی ساده در گوگل، فعالیترین آن‌ها را به شما نشان خواهد داد.
\subsection*{\lr{IRC}}
یک خاطره برای همه گیک‌های قدیمی و هنوز هم اصلی‌ترین کانال ارتباطی حرفه‌ای با یکدیگر. آی.آر.سی. یا اینترنت ریلی چت، یک جور اتاق گفتگو است که در آن می‌توانید با کسانی که در همان لحظه در همان اتاق هستند صحبت کنید. یک نفر همزمان می‌تواند درون اتاق‌های مختلفی از سرورهای مختلف باشد و معمولا هم سریعترین و حرفه‌ای ترین جواب در این اتاق‌ها گرفته می‌شود.
آی.آر.سی. فرهنگ خاص خودش را دارد. در آنجا نباید قبل از اجازه گرفتن از کسی، به او پیام خصوصی بدهید. نباید چیزی بیشتر از دو سه خط را در اتاقی عمومی پیست کنید و به هیچ وجه نباید از فونت های عجیب و غریب و زبان‌های محلی و ... استفاده کنید. در آی.آر.سی. مثل هر جای دیگر مودب باشید و بدانید که اگر کسی به شما کمک می‌کند یک داوطلب بدون چشمداشت است و شما هم باید در حد توان به دیگران کمک کنید.

برای اتصال به یک سرور آی.آر.سی. باید از برنامه‌های ویژه اینکار استفاده کنید.

در ایران این کانال‌های مشهور آی.آر.سی. فعال هستند:
\begin{itemize}
\item تکنوتاکس
\end{itemize}
