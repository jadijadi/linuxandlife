\section{رزومه}
هرچقدر هم که با یک نفر نزدیک باشید یا هر چقدر که خودتان را برای کاری مناسب بدانید، اولین ریپلای به ایمیل درخواست کار این است «رزومه»! نوشتن رزومه در ابتدا کاری بسیار سخت است. اگر هیچ رزومه ای ندارید بهتر است درست بعد از خواندن این متن، یک رزومه برای خودتان دست و پا کنید چون در دنیای حرفه‌ای کار، رزومه شما نه فقط معرف که اولین قدم در گرفتن هر نوع کاری است. رزومه بهتر است در یک فرمت مرسوم و طبیعی مثل برنامه‌های صفحه پرداز آفیس، فایل‌های مارک داون، تخ و ... نوشته شود و در هنگام ارائه به هر کس به پی‌.دی.اف. تبدیل شود. نکات زیر اصلی‌ترین چیزهایی است که در دنیای رزومه نویسی به نظر من می‌رسد. البته مشخص است که بعد از نکات بدیهی‌ای مانند ذکر روش تماس و آوردن تحصیلات و سابقه کاری از جدید به قدیم.
\subsection*{اطلاعات بی‌ربط ننویسید}
این موضوع را بالاتر از همه آوردم چون معمولا در بالای رزومه اطلاعات بی‌ربط زیاد مشاهده می‌شود. در یک رزومه حرفه‌ای در سطح جهان حتی اشاره به سن و جنس هم بی‌ربط به حساب می‌آید چون شرکت‌ها حق ندارند بر اساس سن افراد، در مورد استخدام آن‌ها تصمیم بگیرند چه برسد به وضعیت ازدواج و سربازی و دین و قد و اسم پدر و شماره ملی و غیره. بخش اول رزومه باید معرف هویت شما به شکل کلی و روش‌های تماس با شما باشد. انتخاب اینکه به این بخش عکس اضافه کنید یا خیر با شماست.

خوب است درست زیر بخش اطلاعات فردی، توضیح مختصری در مورد خودتان و اینکه چرا به این شغل خاص علاقمند هستید ذکر کنید.
\subsection*{آپدیت کنید}
همانطور که از نکته قبلی استنتاج می‌شود، رزومه شما بهتر است برای هر درخواست کار آپدیت شود. اگر من برای شغل «مدیر بخش مونیتورینگ» یک شرکت اپلای می‌کنم، بهتر است تجربیاتی مثل کار کردن در ساپورت یا تخصص در سیستم نگیوس پر رنگ تر باشد و مهارت‌های برنامه‌نویسی‌ام کمرنگ‌تر شود. 

در مقابل اگر علاقمند شدم برای رفتن به شرکت فلان، نقش «تحلیل‌گر ارشد» را بپذیرم، سریعا تجربه‌ام در کار در بخش پشتیبانی لایه دو را کم‌رنگ می‌کنم و تاکیدم را روی مشارکت در توسعه سیستم نرم‌افزاری فلان بانک می‌گذارم.

این سرفصل می‌تواند یک معنای دیگر هم داشته باشد. حتی در دورانی که دنبال شغل نیستید هم رزومه‌تان را آپدیت کنید. اگر پوزیشن شما تغییر کرد، مدرک جدیدی گرفتید، مهارتی کسب کردید که نیاز به ذکر دارد و غیره همان موقع در فایل اصلی رزومه این را اضافه کنید تا در آینده راحت باشید.

\subsection*{فقط لینک ندهید}
هر چقدر هم که در جاهای مشهوری کار کرده باشید یا سایت‌های جالبی طراحی کرده باشید، کمتر پیش می‌آید که مسوول بررسی رزومه آدرس سایت‌هایی که کار کرده‌اید را از روی کاغذ در کامپیوتر تایپ کند یا حتی سعی کند بفهمد منظورم من از ذکر اینکه مدرک مهندسی مخابرات‌ام را از دانشگاه
\lr{KNTU}
گرفته‌ام، همان دانشگاه خواجه نصیرالدین طوسی است. بسیار خوب است که جلوی اسم شرکت‌ها، پروژه‌ها،‌ دانشگاه‌ها، سایت‌ها و موارد مشابه چند کلمه توضیح اضافه کنید. مثلا بگویید «تحلیل‌گر در شرکت ایمپک (لهستانی و ارائه دهنده نرم‌افزارهای انتقال پول در شبکه‌های موبایل)».
\subsection*{خلاصه و خوانا بنویسید}
حتما خوب است بخش‌هایی از رزومه با جملات کامل نوشته شده باشند اما در حالت‌های مرسوم اکثریت رزومه به شکل فهرست‌های عددی یا بولت‌لیست و جدول است که بتوان آن را به راحتی مرور کرد. در عین حال لازم نیست تک تک چیزهایی که بلد هستید را ذکر کنید. مثلا من دیگر در رزومه‌ام نمی‌گویم
\lr{HTML}
بلد هستم چون هم از حوزه کاری من خارج است و هم باعث می‌شود چیزهای دیگر به اندازه کافی به چشم نیایند. در سرتیتر بعدی، مفصل‌تر به این موضوع می‌پردازم.
\subsection*{سطح مهارت‌هات را صادقانه بیان کنید}
هر کس مقدار عظیمی چیز بلد است. اگر مدرک نرم افزار دارید بدون شک حداقل زمانی در مورد برنامه نویسی، اسمبلی، معماری کامپیوتر، دیتابیس، ساختار داده و ... چیزهایی بلد بوده‌اید. در یک رزومه حرفه‌ای لازم نیست به تک تک این موارد اشاره کنید. من بارها رزومه‌هایی دریافت می‌کنم که نویسنده‌اش مدعی است «سی، جاوا، سی پلاس پلاس، سی شارپ، جاوااسکریپت، نود، پی اچ پی، پایتون، و این روزها گو، روبی، بیسیک و پاسکال» را می‌داند. شخصا موقعی که صحبت از استخدام یک جاوا یا پی اچ پی کار حرفه‌ای است، هیچ توجهی به چنین رزومه‌ای نمی‌کنم. فراموش نکنید که «برنامه نویسی چیزی بیشتر از دانستن دستور یک زبان است» و برای هر درخواست رزومه‌تان را آپدیت کنید. همچنین لازم است جلوی هر مهارت (از زبان گفتاری تا زبان برنامه نویسی) میزان مهارت‌تان را به شیوه‌ای ذکر کنید.
\begin{mybox}
البته بعضی‌ها معتقد هستند اشاره به همه جزییات رزومه بهتری تحویل می‌دهد. بخصوص در مواردی که یک ماشین در مرحله اول رزومه‌ها را بررسی می‌کند و مثلا ممکن است هر کسی که عبارت
\lr{Office}
در رزومه‌اش نیست، به کلی کنار گذاشته شود. من با این سبک موافق نیستم و انتخاب صد در صد با شماست.
\end{mybox}
\subsection*{نقش خودتان در بخش‌های مختلف را شرح دهید}
فرض کن من در رزومه نوشته باشم «معمار سیستم، شرکت آتی‌تل، ۱۳۹۴ تا امروز». آیا ایده‌ای دارید که من در این شرکت چکار می‌کنم؟ در هر مورد لازم است حتی در یک پارگراف کوتاه در زیر هر سابقه شغلی، توضیح بدهید که دقیقا چه نقشی داشتید، روی کدام پروژه‌ها کار کرده‌اید و خروجی‌ آن‌ها چه بوده. توضیح صادقانه، ساده و بدون اغراق این مساله چیزی است که تقریبا هر کارفرمایی را جذب خواهد کرد و باعث خواهد شد هر کسی که رزومه شما را ببینند، تا این بخش‌ها را کامل نخوانده کاغذ را زمین نگذارد.
\subsection*{ساده و بدون غلط بنویسید}
به روز نگه داشتن و مدیریت یک رزومه فارسی و یک رزومه انگلیسی کار راحتی نیست. به همین دلیل من شخصا فقط یک رزومه ساده انگلیسی دارم. البته در مواردی ممکن است رزومه انگلیسی «کلاس گذاشتن» حساب شود یا بخش مدیریت منابع انسانی شرکت از شما رزومه فارسی بخواهد. این انتخاب شماست ولی هر کدام را که انتخاب می کنید رزومه
\textbf{باید}
بدون اشتباه و سلیس باشد. اگر انگلیسی می‌نویسید لازم نیست پیچیده بنویسید. از کلمات و جملات ساده استفاده کنید و در مرور زمان، جملات را اصلاح کنید.
\subsection*{در رزومه هکر نباشید}
بعله! حتی اگر بزرگترین هکر هستید حداقل در رزومه هکر نباشید. البته من هم هکر بوده‌ام و در بخش زبان‌ها در کنار فارسی و انگلیسی و ترکی آذری از سی هم نام می‌بردم ولی حالا کسی که رزومه‌اش را با فایل متن خالص می‌فرستد و زیرش ذکر می‌کند که این رزومه در
\lr{emacs}
نوشته شده و بالایش یک اسکی آرت هکر گلایدر می‌گذارد را جدی نمی‌گیرم. نه از این نظر که حتما بی‌سواد است بلکه از این نظر که در یک کار تیمی افرادی لازم هستند که با اجتماع اطراف کنار می‌آیند.
\begin{mybox}
سوء تفاهم نشود! تکست محض و گیک‌بازی و آوردن گیک کد هیچ اشکالی ندارد. چیزی که مشکل دارد به رخ کشیدن این چیزها است.
\end{mybox}
\subsection*{مرتبط بنویسید}
رزومه نباید خیلی طولانی باشد. بعضی‌ها حتی می‌گویند اگر حرفه‌ای هستید بیشتر از یک صفحه ننویسید. اطلاعات مختلف برای موقعیت‌های مختلف ارزش‌های متفاوت دارند. اگر برای یک مرکز تحقیقاتی رزومه می‌فرستید معلوم است که تک تک مقاله‌هایتان مهم است ولی اگر برای یک شرکت برنامه‌نویسی اپلای کرده‌اید، مهمترین بخش رزومه مشارکت‌های شما در پروژه‌ها (و بخصوص پروژه‌های‌ آزاد جهانی) است. من هیچ وقت صفحات سه و چهار و پنج رزومه‌ام را با فهرست همه جاهایی که سخنرانی کرده ام پر نمی‌کنم و فقط اگر در شغل خاصی لازم باشد ممکن است در متن پاراگراف اول اشاره کنم که در جامعه لینوکس سخنران هستم. اگر تازه از دانشگاه بیرون آمده‌اید اصلا لازم نیست تلاش کنید رزومه‌تان را به دو صفحه برسانید و ذکر تحصیلات و مشارکت‌های احتمالی در پروژه‌ها و حوزه‌های علاقمندی و دانش کافی است. یک رزومه خوب معمولا از یکی دو صفحه بیشتر نیست.
\subsection*{کمی شخصی باشید}
و البته معنی سرتیتر قبلی این نیست که قرتی بازی‌های شخصی‌تان را کلا فراموش کنید. علاقمندی‌های شخصی را بنویسید. اگر کار جالبی به ذهنتان می‌رسد انجام دهید ولی یادتان باشد که این‌ها بخشی از ورودی‌هایی هستند که برداشت دیگران از شما را شکل می‌دهند. این مساله هم هست که برای مثال مدیر آی تی لازم است بتواند رزومه شما را به بخش منابع انسانی بفرستد پس موضوعات بیش از حد غیرعرف ممکن است علی‌رغم جالب بودن از نظر شما و حتی مدیر، از نظر مدیر منابع انسانی کارها را سخت کند.
\subsection*{شروع کنید}
مهمترین قدم برای داشتن یک رزومه قابل قبول، شروع کردن به ساختن آن است. ما گرایشی داریم که همه چیز را به یک زمان پرفکت موکول کنیم و این کاملا اشتباه است. همین حالا یک ادیتور باز کنید و درفت اول رزومه‌تان را بنویسید. انتخاب ابزار یا یادگیری ابزار جدید اگر بیشتر از پنج دقیقه طول بکشد ضرر است. متن زیر را در یک فایل کپی کنید و قدم اول نوشتن رزومه برداشته شده. انتظار می‌رود با نیم ساعت کار، رزومه قابل قبولی آماده کرده باشید.
\newpage
\begin{latin}
\begin{mybox}
Name:

Family Name:

Nationality:

Applying for:

======================================================

A paragraph about me and why I'm applying for this position.

======================================================

Work History:

======================================================

Knowlege Areas:

Programming

(a table)

Technologies

(a table)

Languages

( a table)

======================================================

Education History:

======================================================

Some personal touch here. Your hobbies, interests, geek code, ... 
whatever you like.

\end{mybox}
\end{latin}
