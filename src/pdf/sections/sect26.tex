\section{آیا شرکت خوبی هست که قدر من رو بدونه}
این مطلب مدت ها است که قراره نوشته بشه. چون مدت ها قبل یک دوست خیلی عزیزم ازم پرسیده:
\begin{mybox}
من سالهاست که وبلاگت رو می خونم و از طرفدارای پر پر و پا قرصشم. طرز نوشتن و بیانت حرف نداره. خیلی وقت ها پیش میاد که بعد از هفته ها یا حتی ماه ها برمیگردم و بعضی از مطالبت رو می خونم.
\href{http://jadi.net/2011/04/\%DA\%86\%DB\%8C\%D8\%B2\%D9\%87\%D8\%A7\%DB\%8C\%DB\%8C-\%DA\%A9\%D9\%87-\%D8\%AF\%D8\%B1-\%D9\%85\%D8\%AF\%D8\%B1\%D8\%B3\%D9\%87-\%DB\%8C\%D8\%A7\%D8\%AF-\%D9\%86\%D9\%85\%DB\%8C-\%D8\%AF\%D9\%87\%D9\%86\%D8\%AF/}{مثل پست «چیزهایی که در مدرسه یاد نمی دهند» که دو باری خونده بودمش و برای بار سوم امروز خوندم}
...

سوالی که برام پیش اومده اینه که آیا واقعا باید هر روز به خودت اون قوانین رو تکرار کنی و بمونی و بجنگی به قیمت به تحلیل رفتن زمان، سلامت روح و جسم و توانایی هایی که می تونی شکوفاشون کنی اما فرصتی برات نیست، یا که نه، شرایط مطلوب تری هم یه جایی ممکنه پیدا بشه؟
تو تجربه ای که در کار با شرکت های خارجی دارم، دستم اومده که همچین جاهایی هستن که قدر بدونن و احترام بذارن و البته به نظرم شاید بی عدالتی هم نباشه اگر وقتی کاری رو درست انجام نمی دی، رفتار خوب دریافت نکنی، اما وقتی کارت به بهترین نحو ممکن انجام می شه، آیا بازم باید رفتار زشت و بد ببینی؟
\end{mybox}
متاسفانه جواب من خیلی دلگرم کننده نیست. شرکت ها بهتر و بدتر دارن ولی چیزی به اسم شرکت قدردان خوب مهربون باشعور قابل وفاداری دوستانه فلان فلان وجود نداره. دلیلش هم ساده است: سرمایه داری. هدف سرمایه داری شکوفا کردن انسان ها نیست بلکه کسب سود بیشتره.

ما توی عصری توی کره‌ای زندگی می کنیم که سیستم تولیدش سرمایه داری است. شاید کشورها اسم خودشون رو بذارن «جمهوری خلق» یا «جمهوری اسلامی» یا هر چیز دیگه ولی در نهایت چیزی که کشورها و در سطح بالاتری کل سیاره رو می چرخونه سرمایه داری است. سرمایه داری یعنی کسی که سرمایه داره قوانینی رو وضع می کنه برای بیشتر کردن سرمایه اش و اگر هم این وسط یکی دلش برای بقیه بسوزه و سعی کنه مهربونتر باشه یا مثلا خدمات بیشتری بده یا کار کمتری بکشه یا هر چی، سریعا توسط شرکت هایی (یا در سطح کلان بخونین کشورهایی) که این «سوسول بازی»ها رو ندارن خورده می‌شه. مثلا شرکت اروپایی فنلاندی من که توش شخصیت آدم ها مهم بود و انسان ها حق داشتن در ابعادی که دوست دارن توش پیشرفت کنن و سفرهاشون کاملا راحت و نسبتا لوکس باشه و ... سریعا توسط شرکت چینی رقیب که توش همه باید مثل سرباز کار کنن وگرنه با یکی از اون یک میلیارد و دویست میلیون و خورده ای (که رقم یکان اون خورده‌ای اش برابر جمعیت فنلاند است) جایگزین می‌شن، تهدید می شه و مجبوره اون خدماتش رو قطع کنه.

البته نه اینکه همه جا مثل هم باشه و همه جا وحشتناک باشه و اینها. ولی یکسری توهم رو نباید داشته باشیم. مثلا این توهم که در خارج کار راحت تره و توی ایران ما خیلی زحمت می کشیم یا مثلا این توهم ریشه ای تر که می رم یک شرکتی و خیلی مهم می شم و توش به من احترام میذارن و ارزش کارم رو درک می کنن. مشکل این نیست که شرکت ها نفهم هستن، مهم اینه که هدف شرکت ها با هدف ما فرق داره. شرکت ها دنبال حداکثر کردن سودشون هستن و ابتکار و خلاقیت و ... شما معمولا هنر خیلی بزرگی در این جریان بازی نمی کنه. همونطور که گفتم بعضی شرکت ها بهتر هستن چون بهتر میفهمن که یک کارمند راضی می تونه سود بهتری برسونه یا مثلا شرایط کاری بهتر باعث جذب نیروهای بهتر می شه که در نهایت سود بهتری می دن یا مثلا با شعورترهاشون شروع می کنن به جایگزین کردن سیستم هرمی با سیستم‌های ماتریسی که توش شما یک رییس ندارین که بتونه شما رو ناراحت کنه (دو تا دارین که تا حدی با هم تضاد منافع دارن و در نتیجه شما راحت تر زندگی می کنین).

اینها شرکت ها رو با هم متفاوت می کنن. به قول سوفیا لورن «پول خوشبختی نمی آره ولی من ترجیح می دم توی یک کادیلاک گریه کنم تا توی یک فولکس». شرکت ها هم با هم فرق دارن. شرکت خیلی جذابی وجود نداره که به شما هم حقوق بده هم توش کاملا خوشحال باشین ولی معلومه که بدون شک بعضی شرکت ها از بعضی شرکت ها بهتر یا با شعورتر هستن اما هیچ تضمینی نیست که در دوران بد اقتصادی، لوس بازی‌های اروپایی رو بذارن کنار تا بتونن با رقبایی که مثل ماشین جنگی کار می کنن، رقابت کنن.

توصیه من؟ اگر واقعا دنبال جایی برای خلاقیت یا پیشرفت چند بعدی یا پولدارشدن یا اینجور چیزها هستین حتما برین سراغ شرکت خودتون یا شرکت‌هایی اونقدر کوچیک که با ورود بهش می تونین صاحبش باشین نه حقوق بگیرش. در غیراینصورت امید بیخودی رو قطع کنین و بدونین که هیچ جای دنیا بر اساس سیستم «قدر دانی از کار خوب»‌ کار نمی کنه. اگر واقعا قرار بود شما برین جایی و خوشحال باشین که دیگه بهتون حقوق نمی دادن (: حالا ممکنه بعضی مزایای خوب وجود داشته باشه یا در دوره هایی از چیز یاد گرفتن یا هر چیز دیگه شاد باشین ولی در کل دلیل اینکه این شرکت ها به آدم ها حقوق می دن اینه که کسی حاضر نیست اون کارهاشون رو داوطلبانه انجام بده. امید بیخودی رو قطع کنین و بدونین که با اینکه شرکت ها خوب و بد دارن ولی همه شون بر اساس منطق «سود»‌ کار می کنن و در یک دنیای پر از رقابت.

خودتون رو هم گول نزنین. نمی تونین یک شرکت خوب شاد درست کنین چون شرکت های خشن بیشتر-از-شما-سود-محور بهتون اجازه نفس کشیدن طولانی نخواهند داد.

اما چاره چیه؟ توصیه بودایی ها رو جدی بگیرین «خواستن رنج است». اگر با این دید می رین به یک شرکت که خیلی باشعور باشه، اشتباه میکنین و چیزی رو می خواین که باعث رنج خواهد شد. در مقابل بدونین که وقتی به جایی می رین برای منافع خودتون رفتین (از چیز یاد گرفتن تا سفر رفتن تا حقوق سر ماه تا ...) و به توصیه اولین مدیرخط من آقای شهپر هم گوش بدین: «غر نزنین خودتون رو هم با بقیه مقایسه نکنین. تا لحظه ای که حس می کنین کار کردن در اینجا راتون می صرفه اینجا باشین و اگر فکر کردین به نفعتون نیست، برین. خیلی ساده». موقع پیدا کردن شرکت به چیزهایی که واقعا براتون مهمه (مثلا برای من سفر، دوست خوب، محیط کار جالب، رزومه جذاب در آینده و ... مهمتر از حتی حقوق است) توجه کنین و سعی کنین به جایی برین که برای آینده به دردتون بخوره. حس نکنین که بهترین جا رو پیدا کردین و گول زرق و برق سرمایه داری رو نخورین چون پشت ظاهر جذاب بهترین شرکت‌ها، منطق خشک و خشن سود است که کار می کنه\footnote{\href{http://jadi.net2012/08/sherkate-khoob-ham-darim/)}{برای بحث به اینجا مراجعه کنین\lr{jadi.net}}}.
