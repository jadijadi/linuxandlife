\section{استفاده از لینوکس با دیسک زنده}
در زمان قدیم (و هنوز هم در مورد سیستم‌عامل‌هایی مثل ویندوز و مک) اگر کسی بخواهد از سیستم عامل استفاده کند، می‌بایست آن را نصب کند. اما چرا؟ مگر «بوت شدن» چیزی بیشتر از لود شدن کرنل و بعد اجرا شدن چند برنامه (مثلا محیط گرافیکی) است؟ این دقیقا سوالی بود که ایگدراسیل لینوکس هم در سال ۱۹۹۲ از خودش پرسید. ایگدراسیل در افسانه‌های نورث، درختی عظیم ست که نه جهان در سر شاخه‌های آن قرار گرفته‌اند. لینوکس ایگدراسیل هم سعی می‌کرد سیستم عاملی باشد که بشود آن را فقط از روی سی دی بوت کرد و سپس از سرشاخه‌های آن که برنامه‌های آزاد دیگر بودند، استفاده کرد.

اما پروژه ایگدراسیل شکست خورد و عرضه آن در سال ۹۵، متوقف شد. یکی از اصلی‌ترین دلایل شکست، سرعت کم درایوهای سی دی آن زمان بود. مقهوم دیسک زنده لینوکس از زمان خود بسیار جلوتر بود و به همین دلیل مجبور شد تا سال ۱۹۹۸ و افزایش سرعت سی‌دی‌ درایوها و حافظه کامپیوترها منتظر بماند.

در ۱۹۹۸ بود که دمو لینوکس به عنوان اولین نسخه عملی یک لینوکس پا به عرصه وجود گذاشت. این نسخه می‌توانست تنها با گذاشتن یک سی دی در درایو و بدون هیچگونه تنظیمی، یک پی سی را در محیط گرافیکی بوت کند و با استفاده از سیستم فایل فشرده‌اش، امکان اجرا کردن صدها برنامه (از جمله استارآفیس) را بدون هیچ تغییری در هارد دیسک فراهم کند. این توزیع تا سال ۲۰۰۲ فعال بود و در آخرین نسخه‌ها امکان نصب بر روی هارددیسک هم به آن اضافه شده بود و این دقیقا چیزی است که راه را به اکثر توزیع‌های حال حاضر جهان نشان داده.

اما قبل از تمام کردن بحث تاریخی، باید ذکری هم از ناپیکس بکنیم. مشهورترین نسخه لینوکس زنده و توزیعی که تقریبا همه آن را به عنوان اولین دیسک زنده‌ای که در تمام جهان استفاده شد، می‌شناسند. ناپیکس توزیعی مبتنی بر دبیان بود که در ۲۰۰۳ رسما عرضه شد. این توزیع می‌توانست به طور گرافیکی یک سیستم را بوت کند و سپس به عنوان یک سیستم عامل برای نجات سیستم‌های مشکل پیدا کرده یا به عنوان یک سیستم عامل مستقل استفاده شود. بوت زنده ناپیکس،‌ آینده تمام توزیع‌های لینوکس بود.

در حال حاضر تقریبا تمام توزیع‌های مشهور لینوکس، نسخه دیسک زنده (\lr{Live disk}) هم دارند و کار تا جایی پیش رفته که می‌شود گفت اکثر توزیع‌های دسکتاپ، اصولا نسخه غیرزنده ندارند. مثلا در مورد اوبونتو، شما همیشه یک فایل \lr{iso} دانلود می‌کنید که بعد از نوشتن آن روی دیسک (به شیوه گفته شده در تهیه یک نسخه از لینوکس) به یک دیسک زنده تبدیل می‌شود. کافی است این سی دی را در درایو یک پی سی که برای بوت از روی سی دی تنظیم شده (در بایوس) بگذارید و کامپیوتر را روشن کنید. مطمئنا به خاطر سرعت کم سی‌دی نسبت به هارد دیسک، بوت شدن کندتر از حالت طبیعی خواهد بود ولی بعد از یکی دو دقیقه، بدون هیچ تغییری در هارد دیسک کامپیوتر سیستم با لینوکس مورد نظر بوت خواهد شد. مثلا اگر از مندریوا استفاده کرده باشید، صفحه چیزی شبیه به این خواهد بود:
\begin{figure}[h]
	\includegraphics[scale=1,width=\textwidth]{../files/images/ubuntu_live_cd.png}
\end{figure}


در این وضعیت شما می‌توانید برنامه‌های موجود را نصب و استفاده کنید، به سخت افزارها دسترسی داشته باشید، در اینترنت گشت بزنید، چت کنید و خیلی کارهای دیگر بدون اینکه به سیستم عامل اصلی نصب شده روی هارد دیسک صدمه‌ای بخورد. دیسک زنده کاربردهای متنوعی دارد. از جمله:
\begin{itemize}
	\item انجام کارهایی که در سیستم عامل خاصی می‌شود انجام داد: مثلا بازیابی اطلاعات هارددیسک یا پاک کردن ویروس‌های یک درایو ویندوزی با استفاده از دیسک زنده لینوکس.
	\item آماده شدن برای نصب: بسیاری از توزیع‌های جدید برای نصب شدن از نسخه زنده استفاده می‌کنند.
	\item بررسی سازگاری سخت افزاری: شما می‌توانید قبل از نصب هر نسخه‌ای از لینوکس، اول آن را به شکل زنده بوت کنید تا مطمئن شوید که با سخت افزار شما مشکل سازگاری ندارد.
	\item آزمایش نرم‌افزارها: مثلا برای تست نسخه جدید مسنجر امپاتی، نیازی نیست حتما آن را نصب کنید بلکه می‌توانید توسط یک دیسک زنده که از این نسخه جدید استفاده می‌کند، آن را در حال کار ببینید و تست کنید.
\end{itemize}
ذکر این نکته هم مهم است که این روزها یو.اس.بی.های زنده هم وارد بازار شده‌اند. در این حالت به جای سی.دی. از یک درایو یو.اس.بی. (مثلا کول دیسک) استفاده می‌شود که درست مثل یک دیسک زنده قابلیت بوت کردن سیستم را دارد اما با یک مزیت مهم: قابل نوشتن است. وقتی از درایو یو.اس.بی. به جای سی دی استفاده می‌شود، تغییرات شما روی سیستم عامل قابل ذخیره شدن روی یو.اس.بی. خواهند بود و این می‌تواند برای کسانی که به شکل حرفه‌ای برای کار از دیسک‌های زنده استفاده می‌کنند، کمک بزرگی باشد.

بعد از اینکه استفاده، آزمایش یا هر چیز دیگری که باعث شده از دیسک زنده استفاده کنید به پایان رسید، کافی است از منو، ری استارت یا شات‌داون را انتخاب کنید و پس از پایان شات‌داون و بیرون آوردن سی دی از درایو، کامپیوتر را دوباره روشن کنید و همه چیز را در وضعیت قدیمی تحویل بگیرید.
