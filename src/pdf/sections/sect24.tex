\section{آیا به دانشگاه بروم}
این سوال رو زیاد می پرسن. راستش رو بخواین من هم زمان کنکورم بود یکی از 
\lr{Day Dream}
هام این بود که بیخیال کنکور و دانشگاه بشم و برم سراغ شرکت باز کردن (دقیقا نمی دونستم چی! فقط اسم شرکت رو می گفتم به عنوان یک شغل) یا مدرک های لینوکس گرفتن یا هر راه فرار دیگه ای برای خلاصی از کنکور.

\textbf{صادق باشیم:}
در بسیاری از مواقع، اینها راه فرار هستن نه نقشه واقعی. متاسفانه رفتن به دانشگاه فعلا معقول ترین مسیر پیشرفته؛ بخصوص توی ایران. دلایلش هم خیلی ساده است. به چند تاش اشاره می کنم.

\textbf{مهمترین دلیل}
اینه که دوران دانشجویی دوران خیلی خاصیه. نه خانواده از شما انتظار خاصی داره نه جامعه. می تونین هر جینگولک بازی ای که می خواین رو در بیارین (از بلند کردن مو گرفته تا انقلاب کردن تا نقد تا فاز هنری گرفتن تا سفر شمال تا ...) و مدعی بشین که «دانشجو» هستین (: دوران تحت کنترل کامل خانواده بودن گذشته و دوران مسوولیت مالی داشتن هم هنوز نرسیده. این دوران خاص رو از دست ندین

\textbf{دومین دلیل}
اینه که دانشگاه اولین محیط مختلط است که بالاخره حکومت ما -کمابیش- قبول می کنه دو نفر آدم حق دارن کنار هم باشن و با هم حرف بزنن. این رو هم الکی از خودتون نگیرین. زمین تا آسمون با پارتی مختلط و اینها متفاوت است. یک تعامل واقعی با جنس مقابل.

\textbf{بعدش اینکه}
در صورت علاقمندی به مهاجرت مدرک دانشگاه بسیار بسیار کمک کننده است. بدون مدارک دانشگاهی تقریبا مهاجرت غیرممکن می شه. البته من طرفدار موندن تو ایران هستم ولی به هرحال اکثریت دنبال مهاجرت هستن و دانشگاه یکی از راحت ترین روش هاش.

\textbf{اینم بگم}
که حتی اگر بخواین در خیلی از کشورها کار کنین، اون کشورها معمولا برای دادن ویزای کار ازتون ترجمه مدرک مرتبط دانشگاهی می خوان. حتی اگر شرکت خودتون بخواد شما رو بفرسته افغانستان کار کنین هم سفارت ازتون مدرک مرتبط با کار می خواد. چه برسه به قطر و غیره (:

می‌بینین؟‌ شاید از نظر عقلی بشه گفت که در سطح علمی مدارک معتبر گرفتن بهتر از دانشگاه رفتنه ولی در دنیای واقعی دور و بر ما، دانشگاه اهمیت خیلی خاصی داره که هر استدلالی که «دانشگاه نرم فلان کار رو بکنم»‌ رو تبدیل می کنه به یک بهانه برای فرار از سختی‌های رفتن به دانشگاه. درسته که این روزها ورود به یکسری از دانشگاه ها خیلی سخته و سال های خوبی از زندگی رو حروم می کنه ولی اگر تصمیم می گیریم نریم دانشگاه لازم نیست دنبال بهانه هایی مثل «دانشگاه مفید نیست» قایم بشیم (:‌ خب روراست بگیم حوصله نداریم دو سال جون بکنیم و رقابت کنیم.. در ضمن چیزهایی مثل دانشگاه آزاد رو هم خدا آفریده دیگه… ورود بهشون راحت تره اسمشون هم دانشگاهه و یکسری از موارد بالا رو هم پاسخگو.
\subsection*{مغلطه: بیل گیتس و جابز و زوکربرگ دانشگاه رو تموم نکردن}
نمی خوان بزنم تو ذوقتون ولی شما بیل گیتس و زوکربرگ و جابز نیستین (: اونها در شونزده هفده سالگی مثل یک گیک دنبال درس بودن نه دنبال دلیل برای نرفتن به دانشگاه. همه وارد دانشگاه شدن و بعد وسطش اونو ول کردن. در ضمن میلیون ها نفر دیگه هستن که دانشگاه نرفتن یا دانشگاه رو ول کردن و من و شما اسمشون رو هم نشنیدیم. در اصل این سه تا و بقیه هم‌پالکی‌هاشون هم چون این سه تا بودن دانشگاه رو ول کردن نه اینکه چون دانشگاه رو ول کردن این سه تا شدن (:

بحث این نیست که اگر کسی نره دانشگاه موفق نمی شه یا دانشگاه ملاک موفقیته یا هر چی. بحث اینه که در یک بررسی معقول دانشگاه رفتن هم توی دید، هم توی زندگی شخصی و هم توی زندگی حرفه ای مفیده. اگر کسی نمی خواد بره حق داره نره و بره دنبال راه خودش. خیلی هم عالیه و قابل افتخار اما اگر حس کردین دارین پشت این استدلال قایم می شین تا از یک چیز سخت به اسم کنکور فرار کنین، یک جای کار می لنگه (:
\subsection*{معیار شخصی}
من یکسری معیار شخصی دارم. یکیش اینه که «کجا و چه زمانی چی می گم». اگر دارم سال آخر که باید درس خوندن جدی برای کنکور رو شروع کنم توضیح می دم «دانشگاه خیلی هم خوب نیست» کمی مشکوکه جریان درست همونطور که اگر درست همون موقع که توی آب چاه خونه ام قرآن افتاده فتوا می دم که «خوردن آب از چاهی که توش قرآن افتاده مشکلی نداره» جریان مشکوکه، با خودتون فکر کنین که چرا دقیقا الان که جریان سخت شده مغزم هی داره سعی می کنه متقاعدم که کنه که «اصلا دانشگاه به درد نمی خوره». اگر واقعا معتقدم آدم موفقی هستم چرا الان که این استدلالها هست نرم دانشگاه و وقتی دقیقا برنامه داشتم ازش نیام بیرون؟ به عبارت دیگه هر وقت حرفی زدم باید ببینم «در کجای جهان ایستاده ام» که اینو می گم (: اگر درست پشت در امتحان کنکور دارم توضیح می دم که دانشگاه اصلا خوب نیست کاملا فرق می کنه با حالتی که توی دانشگاه در حال این استدلال باشم\footnote{\href{http://jadi.net2012/09/is-university-good-for-you/)}{برای بحث بیشتر به این مجموعه کامنت مراجعه کنین\lr{jadi.net}}}.
