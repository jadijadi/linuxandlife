\section{قدم های مرسوم بعد از نصب لینوکس دسکتاپ}
گاهی ممکن است یک سایت یا سند یا حتی برنامه به اجبار سعی در استفاده از یک فونت خاص داشته باشد. مثلا به این اسکرین شات نگاه کنید:

\includegraphics[scale=1,width=\textwidth]{../files/images/sina_blog.png}

تیتر صفحه با صفحه کلید عربی (احتمالا در ویندوز) تایپ شده که به جای «{\settextfont{Tahoma}ک}» از «{\settextfont{Tahoma}ك}»  و به جای «{\settextfont{Tahoma}ی}» از «{\settextfont{Tahoma}ي}}» استفاده می‌کند. 
همچنین طراح به اجبار به صفحه گفته از فونت 
\lr{Tahoma}
 استفاده کند که یک فونت ویندوزی است. برای درست دیدن این صفحه در مرحله اول باید از طراح و نویسنده خواست که از استانداردهای فارسی استفاده کنند و در مرحله دوم،‌ برای حل موقتی مشکل (در اصل برای اضافه کردن امکان نمایش فارسی حروف عربی) باید فونت‌های استاندارد ویندوز را به سیستم اضافه کرد.

در اوبونتو بسته‌ای به نام 
\lr{ubuntu-restricted-extras}
 که مجموعه ای است از ابزارهایی که به علت مجوزهای محدود کننده‌ای که شرکت‌های سازنده روی آن ها گذشته‌اند نمی‌توانند روی سی دی اصلی اوبونتو عرضه شوند. بگذارید نگاهی به محتویات آن بیندازیم:
\begin{mybox}
	\flushleft
	
\lr{jadi@jubung:$\sim$\$ aptitude show ubuntu-restricted-extras}
	
\lr{Package: ubuntu-restricted-extras}
	
\lr{New: yes}
	
\lr{State: installed}
	
\lr{Automatically installed: no}
	
\lr{Version: 57}
	
\lr{Priority: optional}
	
\lr{Section: multiverse/metapackages}
	
\lr{Maintainer: Michael Vogt}
	
\lr{<michael.vogt@ubuntu.com>}
	
\lr{Architecture: amd64}
	
\lr{Uncompressed Size: 30.7 k}
	
\lr{Depends: ubuntu-restricted-addons}
	
\lr{Recommends: ttf-mscorefonts-installer, unrar, gstreamer0.10-plugins-bad-multiverse, libavcodec-extra-53}
	
\lr{Conflicts: ubuntu-restricted-extras}

\lr{Description: Commonly used restricted packages for Ubuntu}
	
\lr{This package depends on some commonly used packages in the Ubuntu multiverse repository.}

\lr{Installing this package will pull in support for MP3 playback and decoding, support for various other audio formats (GStreamer plugins), Microsoft fonts, Flash plugin,}
	
\lr{LAME (to create compressed audio files), and DVD playback.}
	
\lr{Please note that this does not install libdvdcss2, and will not let you play encrypted DVDs. For more information, see}
	
\lr{https://help.ubuntu.com/community/RestrictedFormats /PlayingDVDs}

	
\lr{Please also note that packages from multiverse are restricted by copyright or legal issues in some countries. See http://www.ubuntu.comubuntu/licensing for more information.}

	
\end{mybox}
آنطور که این دستور می‌گوید، این بسته حاوی پشتیبانی از فرمت 
\lr{mp3}
 و فرمت‌های صوتی تصویری دیگر، پشتیبانی از فونت‌های مایکروسافت، فلش و پخش دی وی دی است. پس احتمالا با نصب آن، مشکل ما حل خواهد شد.

\leftline{\lr{sudo apt-get install ubuntu-restricted-extras}}

به عنوان یک نکته اضافی ، خوب است بدانیم که هر کاربر گنو/لینوکس می‌تواند با ساخت پوشه‌ای به اسم 
\lr{.fonts}
 در خانه خودش ( 
\lr{home folder}
 ) و کپی کردن هر فونتی که لازم دارد درون آن و خروج و ورود مجدد به سیستم، به آن فونت‌ها دسترسی داشته باشد. مثلا کپی کردن فایل 
\lr{tahoma.ttf}
 و 
\lr{tahomab.ttf}
 می تواند مشکل وب سایت بالا و بقیه سایت‌هایی که از فونت تاهوما استفاده می‌کنند را حل کند.
 
\textbf{توجه: }
هر فایل که در گنو/لینوکس و سیستم‌های یونیکسی که با نقطه شروع شود،‌ در حالت عادی مخفی است و به کاربر نمایش داده نمی شود. در براوزر فایل گنوم (ناتیلوس) می‌توانید با فشردن 
\lr{Ctrl+H}
 فایل‌های مخفی را ببینید.
