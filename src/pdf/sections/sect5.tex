\section{کاربردهای لینوکس}
\subsection*{سرور}
لینوکس به شکل سنتی به عنوان یک سیستم عامل مناسب برای سرور شناخته شده است. این امر، ارث پدر معنوی لینوکس یعنی یونیکس است. بنا به آخرین گزارش‌های سایت نت‌کرافت، پنجاه درصد بهترین سرویس دهندگان وب از لینوکس بر روی سرورهای خود استفاده می‌کنند\footnote{ از پنجاه درصد باقی مانده سی درصد سهم فری‌بی‌اس‌دی و بیست درصد سهم مایکروسافت است.}.

این سیستم‌عامل همچنین نقش بسیار مهمی در سرورهای مخابراتی دارد و توزیع‌هایی با پشتیبانی تجاری همچون ردهت، درصد زیادی از سیستم عامل سرورها را به خود اختصاص داده‌اند.

کاربرد دیگر لینوکس در دنیای حرفه‌ای، مین‌فریم‌ها و سوپرکامپیوترها هستند. غول دنیای مین‌فریم یعنی آی‌بی‌ام اخیرا اعلام کرده که از سال ۲۰۰۹ به بعد، مین فریم‌هایش را با لینوکس عرضه خواهد کرد. همچنین طبق آخرین آمار سریعترین کامیپوترهای دنیا، از ۵۰۰ سوپرکامپیوتر برتر دنیا،
6.\%88
آن‌ها از لینوکس به عنوان سیستم‌عامل خود استفاده می‌کنند.
\subsection*{سیستم‌های درون‌ساخت}
سیستم‌های لینوکس درون‌ساخت\LTRfootnote{Embedded}
به سیستم‌هایی می‌گویند که لینوکس در آن‌ها به عنوان بخشی از یک ابزار خاص منظوره استفاده شده است. مثلا در یک موبایل، توستر، خودرو، قهوه‌ساز، ساعت یا غیره. تفاوت یک سیستم درون‌ساخت با یک توزیع معمولی لینوکس این است که در وضعیت درون‌ساخت، سخت افزار مورد استفاده کاملا مشخص است و در اکثر موارد هم منابع محدودی دارد. علاوه بر این، معمولا سیستم‌عامل برای اجرای یک برنامه از پیش آماده و ثابت استفاده می‌شود؛ در حالی که از یک توزیع معمولی انتظار می‌رود از سخت‌افزارهای متنوع پشتیبانی کند و بتواند طیف وسیعی از برنامه‌ها را اجرا کند.

در سیستم‌های درون‌ساخت هم لینوکس گزینه مناسبی است. اولا به خاطر قابلیت تغییر ساده، ماژولار، و بعد هم به خاطر هزینه‌های پایین. در حال حاضر لینوکس در تلفن‌های موبایل، انواعی از روترها و فایروال‌های شبکه (به‌ویژه در لینک‌سیس‌های سیسکو سیستمز) و ابزار دیگر استفاده می‌شود. با پیدایش سیستم‌عامل اندروید گوگل، استفاده از لینوکس درون‌ساخت در وسایل ارتباطی شتاب خیلی بیشتری گرفت.
\begin{figure}[h]
\includegraphics[scale=1,width=\textwidth,keepaspectration=true]{../files/images/android_phone.jpg}
\caption*{نمونه‌ای از یک گوشی اچ‌تی‌سی مجهز به لینوکس اندروید}
\end{figure}
\subsection*{دسکتاپ}
لینوکس چند سالی بیشتر نیست که به شکل جدی وارد دنیای دسکتاپ شده است. تقریبا ده سال پیش اگر می‌خواستید لینوکس را روی دسکتاپ یا لپ‌تاپ خود نصب کنید، دردسرهای خیلی زیادی داشتید اما حالا، همه چیز راحت تر است. در حال حاضر تقریبا تمام توزیع‌ها به شکل پیش‌فرض از محیط گرافیکی استفاده می‌کنند (مثال نقض جذاب،‌ آرچ است) و به خاطر استفاده از کرنل‌های جدید، امکان شناسایی طیف بسیار وسیعی از سخت‌افزارها را دارند.

این روزها تقریبا نسخه جدید هر توزیعی را که داشته باشید، انتظار می‌رود بدون مشکل روی دسکتاپ یا لپ‌تاپ شما نصب شود و بدون دردسر به شما اجازه استفاده یا نصب طیف بسیار وسیعی از نرم‌افزارها را بدهد. از برنامه‌های آفیس گرفته تا بازی‌های سه بعدی، پخش کننده‌های موسیقی، چت، و ضبط کننده‌های تصاویر.

در حال حاضر در عمل تمام برنامه‌های عمومی موجود برای ویندوز و مک، نسخه‌های مشابه لینوکسی هم دارند. گاهی این برنامه، دقیقا نسخه مشابهی است که روی لینوکس کامپایل شده (مانند فایرفاکس) یا نسخه‌ی لینوکسی یک برنامه ویندوزی است (مانند اسکایپ) یا برنامه‌ای است که قابلیت کاری مشابهی با برنامه ویندوزی یا مکی دیگری دارد (مانند گیمپ که جایگزین فتوشاپ است).

البته نکته‌ای که نباید فراموش کنید این است که لینوکس ویندوز نیست. نباید انتظار داشته باشید که دقیقا هر چیزی روی ویندوز اتفاق می‌افتاد روی لینوکس هم اتفاق بیافتد. گاهی برنامه‌های تخصصی وجود دارند که مشابه لینوکسی ندارند (مانند یک برنامه‌ی خاص طراحی مدار) و گاهی شکل رابط کاربری یک برنامه کاملا با همکار ویندوزی‌اش فرق می‌کند (مثلا در مورد گیمپ).

همچنین ذات باز و آزاد لینوکس، به افراد اجازه داده تا توزیع‌های ویژه کاربردهای خاص را ایجاد کنند. مثلا توزیعی مثل پارسیکس با هدف پشتیبانی پیش فرض از زبان فارسی ساخته شده یا توزیعی مثل میت تی.وی. به طور خاص برای تبدیل یک کامپیوتر به یک مدیاسنتر توسعه یافته است. این موضوع هم باعث شده درصدی از کاربران خانگی به سراغ لینوکس‌های خاص منظوره کشیده شوند.
\subsection*{کمک به جامعه‌ی بشری و ساختن یک دنیای بهتر}
\includegraphics[scale=1,width=\textwidth,keepaspectration=true]{../files/images/olpc1.png}
پروژه‌ی "یک لپ‌تاپ برای هر کودک\LTRfootnote{\href{http://www.laptop.org}{www.laptop.org 
	}}" 
\lr{One Laptop per Child (OLPC)}
پروژه‌ای است با هدف بالابردن سطح آموزش در فقیرترین کشورهای جهان که توسط چند سازمان غیرانتفاعی دنبال می‌شود. این لپ‌تاپ‌های ارزان‌قیمت و کم‌قدرت (این موسسه اخیراً نوعی تبلت نیز در همین راستا ارائه کرده است) که XO نامیده میشوند مخصوص کودکان ساخته شده‌اند. سیستم عاملی که در این لپ‌تاپ‌ها استفاده می‌شود نوعی لینوکس مبتنی بر توزیع 
\lr{Fedora}
 هست که از رابط کاربری 
\lr{Sugar}
 استفاده میکند.

\includegraphics[scale=1,width=\textwidth,keepaspectration=true]{../files/images/olpc2.jpg}
شاید خوشبین‌ترین افراد نیز تصور اینکه پروژه‌ی گنو/لینوکس و تلاش‌های استالمن، توروالدز و هزاران انسان دیگر روزی بدین شکل برای کمک به جامعه‌ی بشری و گسترش آموزش و پرورش در اینچنین ابعاد جهانی به خدمت گرفته شود، را نداشتند اما امروز این خواسته محقق گشته است. موسسه‌ی 
\lr{OLPC}
 در وب‌سایت خود ماموریت این نهاد را اینگونه شرح می‌دهد:
"آموزش شالوده‌ای است برای انسان کامل، توسعه‌ی اجتماعی، اقتصادی و دموکراتیک. با دسترسی به این نوع ابزار، کودکان در آموزش خود درگیر میشوند، یاد میگیرند، به اشتراک میگذارند و باهم خلق میکنند. آنها به یکدیگر، به جهان و به یک آینده‌ی روشن‌تر ملحق میشوند."
\begin{center}
	\begin{figure}[H] 
		\subfloat{\includegraphics[scale=1,width=0.5\textwidth,keepaspectration=true]{../files/images/olpc4.jpg} 
		} 
		\hfill 
		\subfloat{\includegraphics[scale=1,width=0.5\textwidth,keepaspectration=true]{../files/images/olpc3.jpg}
		} 
	\end{figure}
\end{center}




