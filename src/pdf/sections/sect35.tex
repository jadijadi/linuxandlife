\section{ایجاد انگیزه و تمرکز}
ما کارهای خوبی رو دوست داریم بکنیم. ما به شکل سنتی در شروع هر سال فهرستی کاغذی یا ذهنی از کارهای اون سال داریم. ما می‌دونیم می‌خوایم وقتی بزرگ شدیم چی بشیم و ما می‌دونیم که قدم بعدی خوب در این روز اینه که چی رو یاد بگیریم. اما تقریبا هیچ کدوم از ما فرانسه یاد نمی‌گیریم، قهرمان هک نمی‌شیم، سیکس پک نداریم و ...

بقیه هم این وضعیت رو می‌دونن. باشگاه‌ها می‌دونن که می‌تونن چند برابر ظرفیت ثبت نام کنن چون اکثر کسانی که ثبت نام کردن دو سه جلسه بیشتر به باشگاه نخواهند اومد. سایت‌های تخفیف گروهی و غیره سود می‌کنن چون درصد قابل توجهی از کسانی که پول می‌دن تخفیف‌ها رو می‌خرن حوصله نمی‌کنن برن از تخفیفی که براش پول دادن استفاده کنن.

چرا اینطوریه؟ گفتن اینکه همیشه اینطور می شه چون آدم انگیزه‌اش حفظ نمی‌شه یا پشتکار نداره مثل اینه که بگیم یک خودرو در مسیر متوقف شده چون سرعتش به صفر رسیده! خب سوال اصلی اینه که چرا انگیزه نداریم و چرا پشتکارمون حفظ نمی‌شه. به نظر من اینها اصلی‌ترین عواملی هستن که باعث می شن پشتکار ما در طول مسیر از بین بره و نظرات‌مون تغییر کنه و تمام سال‌های بعدی رو هم مثل سال‌های قبلی بگذرونیم:
\subsection*{پیشرفت سخت است}
فانتزی اینکه ما یک قهرمان باشیم خیلی جذابه ولی مسیر اینکه به قهرمانی برسیم مسیر سخت و پر حوصله‌ای است. برای یک دونده دوی استقامت دائما فکر کردن به اینکه‌ «پس کی می رسم» بدترین کار است. من همیشه نقل می کنم که «آسانسور پیشرفت اختراع نشده و باید قدم به قدم پیش رفت». اگر قراره واقعا یک روز نفر اول فلان چیز باشیم، لازمه که هر روز یک قدم پیش بریم بدون اینکه نگران باشیم که خیلی ها از ما جلوتر هستن. در واقع شما فقط وقتی می تونین در دنیاهای کامپیوتری و خیلی چیزهای دیگه زندگی بهترین باشین که از مسیر لذت ببرین نه از بودن در هدف. درست همونطور که یک کوهنورد باید از کوهنوردی کردن لذت ببره نه از «بودن در قله».
\subsection*{از خودمون نباید جلو بیافتیم}
من در مدرسه تیزهوشان درس می‌دادم و به نظرم بدترین چیز در اونجا این بود که بعضی ها از خودشون جلو می‌افتادن. گروهی بود که به همراهی «مربی‌»هاشون در دبیرستان کارت صوتی کامپیوتر ساخته بودن و خب معلومه که قدم بعدی این آدم نمی تونه «یاد گرفتن الفبای الکترونیک» باشه. اگر شما دارین برای بقیه جلوه‌ای «جلوتر از خودتون» نمایش می‌دین، منطقا هیچ گاه وقت و حوصله برداشتن قدم‌های واقعی اولیه رو نخواهید داشت. برای روشن شدن مساله یک مثال می‌زنم. فرض کنین گروه از دوستان شما تصمیم گرفتن برن استخر و شنا یاد بگیرن. شما هم دوست دارین قهرمان شنا باشین و برای اینکه قافله عقب نمونین وقتی از استخر برگشتن، بهشون می گین که درسته نیومدین استخر ولی شناتون خیلی خوبه. این باعث می شه دیگه هیچ وقت نتونین باهاشون برین استخر و حتی پایه‌های شنا رو یاد بگیرین. اگر قراره در دنیای تکنولوژی پیشرفت کنین باید همیشه سعی کنین زیرپاتون سفت باشه، از خودتون جلو نیافتین و هر چیزی که متوجه نمی‌شین رو به شکل پایه‌ای بخونین و یاد بگیرین. اتفاقا آدم های باسواد اطراف شما کسانی هستن که مقدمات رو خوب بلدن نه کسانی که چیزهای پیشرفته رو خیلی خوب بلدن‌ (:
\subsection*{مسیر حرکت شما باید رو به جلو باشه}
همین الان که دارین این رو می‌خونین به گذشته فکر کنین و ببینین برای پیش اومدن واقعا چه قدم‌هایی برداشتین؟ اگر من سه ساله که فرانسه نخوندم نمی‌تونم هدف سال آینده رو بذارم «سه تا زبان یاد بگیرم». باید در مسیر معقولی آهسته ولی منظم حرکت کرد. هیچ کس نمی‌تونه از فردا یک آدم دیگه بشه! اهداف کوچیکتر انتخاب کنین و‌آروم آروم بهشون برسین.
\subsection*{مواظب سرمایه‌داری خبیث باشین}
برای اینکه سیستم سرمایه داری بچرخه، ما باید دائما یکسری آرزو بخریم و بفروشیم. یکی مشغول فروختن افزایش دهنده سایز است و یکی دیگه مشغول فروختن آپارتمانی در ترکیه که قراره توش زندگی رویایی داشته باشین و یکی دیگه سعی می کنه خدای فلان چیز بشه تا همه بهش احترام بذارن یا دوستش داشته باشن یا پولدار بشه! دنیای واقعی جایی نیست که ما بتونیم به همه آرزوهامون برسیم؛ بخصوص که همین که به اولین آرزو برسین سه تا آرزوی جدید جلوتون سبز می‌شه. در حین نوشتن این من توی یک کافه نشسته‌ام. از اطرافیان می‌پرسیم نظرشون در این باره چیه. اولی صاحب کافه است که روزی آرزوش بوده کافه خودش رو داشته باشه ولی امروز چندان هم شاد نیست. دومی یک دانشجوی دکترا است که بچه و زندگی اش رو داره؛ زمانی که مهندس بوه آرزوش این بوده که روزی دکترای جامعه شناسی باشه ولی هنوز شاد نیست و نفر سوم پزشکه و شرایطش نسبتا مشابه.

واقعا شما هر چقدر هم که پیشرفت کنین بازم حس می کنین موفق نشدین. همین الان من نگران هستم که چرا در تابستان پیش رو سیکس پک ندارم و اگر داشتم بهتر بود و لازمه عاقلانه به خودم بگم که من اصلا اون آدم نیستم (: شما هم مواظب باشین، شاید دارین به اندازه کافی پیشرفت می‌کنین ولی متوجه‌اش نیستین.
\subsection*{شما تعیین کننده همه چیز نیستین}
آدم ها شرایط بسیار متفاوتی دارن. مغز ما با هم فرق می‌کنه و اینکه شما الان علاقمند به تکنولوژی هستین معنی اش این نیست که بهترین در دنیای تکنولوژی خواهید بود. من سال‌ها به شکلی جدی یادگیری و تمرینات فوتبال رو دنبال می‌کردم ولی من اصولا تیپ ورزشکار نیستم. این رو باید قبول داشت. از اونطرف من از شش سالگی با کامپیوترها بزرگ شدم در حالی که یک نفر شاید اولین کامپیوترش رو بعد از اینکه یکسال از کار کردنش توی یک کارگاه گذشت بتونه بخره. شرایط آدم ها فرق می کنه و این ایده که «هر کس تلاش کنه موفق می‌شه» تا حد زیادی غیرواقعی است. مثال پرت کردن کاغذ توی سطل عالیه:

معلمی در سر کلاس از بچه‌ها می‌خواد نفری یک کاغذ بردارن، اسمشون رو روش بنویسن و بعد مچاله‌اش کنن. بعد سطل آشغالی رو پای تخته می‌ذاره و می‌گه هر کس بتونه از جایی که نشسته کاغذش رو داخل سطل بندازه، در زندگی موفق / پولدار / مشهور و ... خواهد بود. نفرات جلویی با درصد موفقیت بیشتری پرت می‌کنن و اصلا حواسشون نیست که عقبی‌ها شانس خیلی کمتری دارن. کاملا می‌شه پذیرفت که کسی که در عقب کلاس بیشتر تلاش و تمرین و پیگیری و ... کنه شانسش بالا می‌ره ولی در نهایت جایی که شما نشستین اولین تعیین کننده است و درصد موفق‌ها در آخر کلاس در پایین‌ترین مقدار ممکن.
این مثال در دنیا هم صادقه. از اونطرف هم شما با یک سیستم بزرگ طرف هستین که سعی می کنه به شما جهت بده. درس‌هایی که بخونین اونجا تعیین می‌شه، اینکه دسترسی شما به اینترنت به چه شکل باشه اونجا تعیین می‌شه و ... حتی در سطح متوسط هم شما با خانواده درگیر هستین، با خواسته‌های دوستان، با پدر و مادر، با عشقی که یکهو پیش می یاد و این تیپ چیزها. پس نقش شخص شما در این انتخاب‌ها اونقدرها هم زیاد نیست و باید حواستون باشه که اگر قراره به چیزی که می‌خواین برسین، باید خیلی جدی تلاش کنین و حاضر باشین با عوامل بیرونی هم بجنگین.
\subsection*{انسان یک پیوستار نیست}
تصور اشتباه ما از انسان یا حتی «خودم» اینه که در یک پیوستار ثابت زندگی می‌کنیم در حالی که باید بگم اینطور نیست. مغز و بدن ما شدیدا تحت تاثیر هورمون‌ها و شرایط بیرونی و درونی است و مغز شمای گرسنه با شمای سیر مغزی متفاوت است و در نتیجه اگر کسی تصمیم‌های جادی خسته از یک شب پر از نوشیدنی و رقص عالی رو با مغز جادی بعد از خوندن کلی خبر تکنولوژی باحال مقایسه کنه به دو موجود کاملا متفاوت می‌رسه. اولی می‌خواد بازم شادی کنه یا بخوابه و دومی می‌خواد هر طور شده خودش رو برسونه به یک میکروفون و رادیو گیک\LTRfootnote{\href{http://jadi.netpodcast}{jadi.net}}
ضبط کنه. مثال دیگه ورزش است. من وقتی ورزش می‌کنم بدنم پر از هورمون‌های لذتی می‌شه که منو مطمئن می‌کنه فردا هم حتما ورزش خواهم کرد ولی وقتی فردا می‌شه اون هورمون‌ها دیگه توی رگ‌های من نیستن و به نظرم عجیب‌ترین کار این می‌یاد که الان که از تخت‌خواب بیرون بیام و بالا پایین بپرم!

اشتباه ما معمولا اینه که فکر می کنیم «من امروز، من فردا هستم» و دقیقا برای همینه که امروز تصمیم‌های مهمی می‌گیریم که انگار روی سنگ حک شدن ولی فردا کلا بیخیال اون تصمیم هستیم و اولویت‌های دیگه‌ای داریم. خوندن رمان و فلسفه برای همین توصیه می شه. این چیزها شاید به شکلی عمیق تر از یکسری هیجان آنی یا تصمیم منطقی بتونن شیوه تفکر مغز رو شکل بدن و باعث بشن در مواقع مختلف نگاه مغز به دنیا ساختار ثابت‌تری داشته باشه و مثلا همیشه بدونه که ادامه دادن یک مسیر نتیجه بهتری از زیگزاگ رفتن می‌ده.
\subsection*{چیزهای ضروری در مقابل چیزهای مهم}
توجه کنید که این حرف‌ها اصلا به این معنی نیست که «امکان نداره» بلکه دقیقا ماجرا برعکسه. آدم‌های موفق تونستن به این عوامل و عوامل مشابه غلبه کنن و به نظر من اگر هدف درست و معقول انتخاب بشه حرکت به سمتش ساده است. چیزی رو انتخاب کنین که براتون مناسبه، واقعا می‌خواینش و از همه مهمتر از مسیر رسیدن بهش هم لذت می‌برین. اگر دنبال شهرت یا پول هستین به هیچ وجه سراغ تکنولوژی و غیره نرین چون مسیر کم‌شانس و سختی است اما اگر واقعا از خود مسیر یاد گرفتن لذت می‌برین، مثل یک کوهنورد قدم به قدم پیش برین و آرزو کنین دیرتر به قله برسین چون خود حرکت است که باید لذت بخش باشه.
