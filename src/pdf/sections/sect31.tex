\section{راهنمای انتخاب زبان برنامه نویسی}
\begin{mdframed}
جادی جان منظور من این بود چه نوع زبان برنامه نویسی از بین اون دوره ها خوبه که دوستم شرکت کنه؟ جاوا؟ اوراکل؟ دات نت؟ نمیخوام آموزشگاه بهم معرفی کنی میخوام یک زبان برنامه نویسی که میدونی الان بیشتر کاربرد و بازار کار داره رو بهم بگی.
\end{mdframed}
ضرب‌المثلی در دنیای برنامه نویسی هست که می گه «اشتباهی که خیلی از برنامه‌نویس‌های تازه کار می کنن اینه که برنامه‌نویسی رو با یاد گرفتن کد نوشتن به یک زبان خاص اشتباه می‌گیرن». پس توصیه اول به این دوستمون اینه که دنبال یاد گرفتن برنامه نویسی باشه نه یاد گرفتن دستور زبان یک زبان خاص.

 اکثر زبان‌ها در پایه به هم شبیه هستن و اگر یک زبون رو درست یاد بگیریم سوییچ کردن یا کد نوشتن به یک زبان هم‌خانواده چندان مشکل نیست. اینه که یاد گرفتن \lr{C} می تونه پایه خوبی برای هر برنامه نویسی باشه و بعدش خوندن کد چند برنامه خوب (که روی گیت هاب به راحتی قابل دسترسی هستن و حتی می تونن به سادگی برنامه ای مثل \lr{yes} در لینوکس باشن).
نکته بعدی اینه که آدم‌ها به شیوه‌های مختلفی چیز یاد می گیرن. در کل به سه شیوه:
\begin{enumerate}
\item خوندن (کتاب، راهنما، ...)
\item آموزش دیدن (کلاس)
\item انعکاسی (دیدن و تکرار کردن)
\item تمرین کردن (پریدن وسط استخر و دست و پا زدن)
\end{enumerate}
که البته مثل هر چیز دیگه‌ای، در دنیای واقعی هر کدوم با ترکیبی از روش‌های بالا به حداکثر یادگیری خودمون می‌رسیم.

\href{http://jadi.net/2011/12/\%da\%86\%d8\%b7\%d9\%88\%d8\%b1\%db\%8c-\%d8\%a7\%d8\%b3\%da\%a9\%d8\%b1\%db\%8c\%d9\%be\%d8\%aa-\%d9\%86\%d9\%88\%db\%8c\%d8\%b3\%db\%8c-\%d8\%a8\%d8\%b4-\%db\%8c\%d8\%a7-\%d9\%87\%d8\%b1-\%da\%86\%db\%8c\%d8\%b2-\%d8\%af\%db\%8c\%da\%af\%d9\%87-\%db\%8c\%d8\%a7/" title="چطوری اسکریپت نویسی بش یا هر چیز دیگه یاد بگیریم؟}{مثلا می دونم که بهترین روش یادگیری ام ترکیبی از خوندن و تمرین کردن است}.
 این اصل ساده در آموزش گاهی به خاطر عادت ۱۲ ساله ما به مدرسه‌های کلاس محور ایرانی کلا فراموش می شه. پس قبل از ثبت نام کلاس یک لحظه باید به خودمون یادآوری کنیم که یاد گرفتن چیزها الزاما نیازمند کلاس نیست و اگر کل راهنماهامون رو از اینترنت بگیریم و کل هزینه کلاس رو خرج خوش گذرونی کنیم ممکنه خیلی بهتر چیز یاد بگیریم.

اما کدوم زبون؟ فرض کنیم دقیقا دنبال یاد گرفتن یک زبون برنامه نویسی هستیم. چه زبونی بهترینه؟ نقل می کنم از
\textbf{سینا}
ی عزیز که
\begin{mdframed}
\centering
تا وقتی اهداف رو ندونیم نمی تونیم مسیر رو ارزیابی کنیم.
\end{mdframed}
هدف ما چیه؟‌ رسیدن به یک شغل مطمئن توی بازار کاری که از همیشه برامون کار هست؟ گرفتن پروژه‌های خاص با پول خیلی زیاد؟ مهاجرت؟ کار کردن توی یک استارتاپ مرتبط با گوشی‌های موبایل؟ اپلای کردن برای گوگل؟

\textbf{هدف}
رو اگر بدونیم تقریبا مشخصه که باید چیکار کنیم. راه بر اساس مقصد قابل تشخیصه و حتی در مواردی انتخاب مقصد به معنی انتخاب راه است.

بذار موضوع انتخاب بهینه زبان برنامه نویسی رو با یک نمودار متقاطع توضیح بدم. اطلاعات ما در جهان یا درباره حوزه‌های شناخته شده برای همه است که تبدیل می شیم به عضوی از یک خیل عظیم (مثلا بعد از چهار سال می شیم مهندس مخابرات فارغ التحصیل از دانشگاه) یا وقت گذاشتیم و در حوزه‌های کمتر شناخته شده برای عموم اطلاعات کسب کردیم (و مثلا شدیم کسی که تجربه و دانش زیادی در مورد چاپ کتاب داره). اولی شغل‌های بیشتری داره ولی معمولا حقوقش کمتره چون کلی آدم دیگه هم هستن که همونها رو بلدن.

 از اونطرف یک بحث دیگه اینه که دانش ما چقدر عام است و چقدر خاص. آیا ما «کلا برنامه نویسی بلدیم» یا «می دونیم چطوری باید برنامه های چند رشته ای (مالتی ترد) نوشت» ؟ آیا ما کلا کتاب چاپ کردن بلدیم یا تخصصمون دقیقا در این است که بدونیم بهترین چسب موجود در بازار برای چسبوندن جلد کتاب به شیرازه، چیه. بذارین این دو تا رو روی نموداری که حرفش رو می زدم نشون بدم:
\begin{figure}[H]
	\begin{center}
		\includegraphics[width=0.8\textwidth]{../files/images/what_programming_lang_to_study.png}
	\end{center}
\end{figure}
با توضیات بالا واضحه که یک برنامه نویس دات نت همیشه حقوق داره و همیشه در شرکت های متوسط کار داره. استرسش برای پیدا کردن کار کمه ولی در عوض در نگه داشتن کار اوضاعش خوب نیست چون هزاران نفر هر سال مجموعه مدرک های
\lr{MCSE}
\lr{MCTS}
 و غیره رو می گیرن و می شن برنامه نویس دات نت. در مقابل به بخشی نگاه کنین که با «پول پروژه‌ای» مشخص شده. ما الان در شرکت دو ماهه دنبال کسی می گردیم که به شکل پروژه‌ای بیاد برای ما سرورهای نود جی اس رو کلاستر و \lr{High Aavailable} کنه و کم اهمیت ترین موضوع در پروسه قرارداد اینه که طرف چقدر پول می خواد. احتمالا طرف با این کار چند روزه می تونه به اندازه چند ماه برنامه نویس دات نت پول در بیاره ولی ظاهرا در کشور عزیز افراد خیلی خیلی کمی هستن که این کار رو بلد باشن چون هم در حوزه ناشناخته است و هم در حوزه تخصصی.

حالا فکر می کنم انتخاب براتون راحت تر باشه. من همیشه در حوزه ناشناخته تخصصی بودم و راستش یک محور دیگه هم در جدول کشیدم: ترکیبش با یک چیز دیگه. مثلا متخصص سیستم عاملی که مخابرات بلده. مدیر پروژه‌ای که لینوکس بلده و ... این خیلی کم پیدا می شه و خیلی هم مورد نیاز نیست ولی اگر کسی شما رو بخواد... واقعا شما رو همه جوره می خواد.

 در عوض ممکنه شما به این نتیجه برسین که یک کار امن و راحت می خواین و در این صورت باید نگاهی به بخش استخدام شرکت ها بندازین و ببینین الان چی بورسه. ممکنه دات نت باشه و ممکنه جاوا باشه یا اگر تصمیمتون این شده که برین سراغ استارتاپ موبایلی شخصی خودتون، معلومه که باید یا \lr{iOS} یاد بگیرین یا \lr{Android Development} که یک جاهایی بین جدول بالا افتاده.

\textbf{توجه:}
جدول رو قارتی کشیدم. اصلا معنی اش این نیست که کاملا اندیشیده و دقیق است. می شه در مورد همه اجزاش حرف زد.

\textbf{جایزه اینکه تا اینجا خوندین}
: اگر هدفتون رفتن به چیزی مثل گوگل است، برین و آگهی‌های شغلی اش رو ببینین و بعد دقیقا می دونین که در طول یکی دو سال آینده باید چی یاد بگیرین و به کدوم پروژه‌ها کمک کنین تا خود گوگل بیاد ازتون خواهش کنه که برین تو شرکتش کار کنین (: این دیدن ترند مهمه. لازمه درک کنیم که در جهان آینده چه چیزهایی دارن رو می یان و براشون آماده باشیم\footnote{مرتبط طنز:\href{http://jadi.net/2008/05/\%D8\%A2\%DB\%8C\%D8\%A7-\%D9\%85\%D9\%88\%D9\%81\%D9\%82\%DB\%8C\%D8\%AA-\%D8\%B2\%D8\%A8\%D8\%A7\%D9\%86\%E2\%80\%8C\%D9\%87\%D8\%A7\%DB\%8C-\%D8\%A8\%D8\%B1\%D9\%86\%D8\%A7\%D9\%85\%D9\%87\%E2\%80\%8C\%D9\%86\%D9\%88\%DB\%8C\%D8\%B3\%DB\%8C-\%D8\%A8\%D8\%A7/}{آیا موفقیت زبان های برنامه نویسی به ریش و سبیل سازندگانش مربوطه؟}}
تا اینجا خوندین؟\footnote{ (: پس شاید دوست داشته باشین
\href{http://jadi.net/2008/09/chegoone-barname-nevis-besham/" title="پادکست اول: چگونه می‌توانیم یک برنامه‌نویس خوب بشویم؟}{این شماره ویژه پادکست در مورد برنامه نویسی رو هم گوش بدین}.}.

\textbf{پی نوشت آخر}
. گفتیم که «اشتباه برنامه نویس های جوان اینه که برنامه نویسی رو با برنامه نویسی به یک زبون خاص اشتباه میگیرن». این یعنی اگر شما برنامه نویسی رو یاد بگیرین می تونین بعدا به هر چیز دیگه سوییچ کنین. رو پایه ها تمرکز کنین و تا وقتی درک می کنین که دارین برنامه نویسی یاد میگیرین، نگران زبون نباشین.

و البته
\href{http://jadi.net/2010/12/\%D9\%85\%D8\%AD\%D8\%A8\%D9\%88\%D8\%A8\%D8\%AA\%D8\%B1\%DB\%8C\%D9\%86-\%D8\%B2\%D8\%A8\%D8\%A7\%D9\%86\%D9\%87\%D8\%A7\%DB\%8C-\%D8\%A8\%D8\%B1\%D9\%86\%D8\%A7\%D9\%85\%D9\%87-\%D9\%86\%D9\%88\%DB\%8C\%D8\%B3\%DB\%8C/}{به فهرست محبوبیت زبان‌های برنامه نویسی در جهان هم نگاهی بندازین}
و اگر هنوز سوالی هست در کامنت ها مطرح کنین تا سعی کنم از برنامه نویس هایی بسیار خوب بخوام جواب هاش رو براتون بگن.
\begin{mdframed}
\href{http://sadeq.ir}{صادق}
توی کامنت‌ها نوشته:
ضمن تایید حرف‌های جادی می‌خواستم برای کسایی که تازه می‌خوان شروع کنند یه چند تا نکته را یادآوری و تاکید کنم:
\begin{enumerate}
	\item سعی کنید با زبانی مثل \lr{python} شروع کنید که هم زود بتونین نتیجه بگیرین و هم عادت‌های خوب کدنویسی براتون نهادینه بشه.
	\item از خوندن و یادگرفتن \lr{Design pattern}ها و \lr{Best practice}ها ولو با زبان دیگه‌ای غیر از زبان تخصصی شما پیاده‌سازی شده غفلت نکنید که موجب پشیمانیست.
	\item  از ابتدای کار عادت کنید برای کدهاتون کامنت مناسب و واضح بنویسید.
	\item کد خوب بخونید (توی پروژه‌های آزاد اغلب کدها خیلی خوبند چون تعداد زیادی توش مشارکت می‌کنند)، توی توسعه پروژه‌های آزاد مشارکت کنید از گزارش باگ گرفته تا نوشتن پلاگین و ...
	\item  سعی کنید ارتباط‌هاتون با آدم‌های متخصص توی زمینه کاریتون زیاد کنید، از فرصت گپ زدن باعاشون استفاده کنید، توی فروم و \lr{irc} فعال باشین نه فقط برای پرسیدن که برای مشارکت و پاسخ دادن.
	\item خوشبین باشید و شاد. کاری را انجام بدین که ازش لذت می‌برین. البته لذت بردن خیلی وقت‌ها به معنی آسون بودن یا پول زیاد درآوردن نیست. در ضمن سعی نکنید تقلید کنید.
	\item وقتی دارین کد می‌خونید واقعا بفهمید چی به چیه. و بد نیست فکر کنید روش بهتری برای پیاده‌سازی اون کد به ذهنتون می‌رسه یا نه.
	\item وقتی تونستید کد بقیه را سریع دیباگ کنید می‌تونید کم‌کم حس کنید که حرفه ای شدین. البته یادگرفتن انتها نداره.
\end{enumerate}
\end{mdframed}
برای بحث بیشتر به \lr{jadi.net} مراجعه کنین\LTRfootnote{\href{http://jadi.net2013/12/kodoom-zaboone-barname-nevisi-behtare/)}{jadi.net}}
