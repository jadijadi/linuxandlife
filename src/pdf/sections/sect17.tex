\section{آشنایی با خط فرمان}
خط فرمان لینوکس که می‌توانیم به آن پوسته (
\lr{Shell}
) هم بگوییم، جایی است که کاربر می‌تواند به شکل مستقیم و از طریق تایپ روی صفحه کلید، دستوراتی را به سیستم بدهد یا فایل‌های تنظیمات را ادیت کند. این صفحه گاهی به این شکل است:
تصویر یک کامند پرامپت ساده
و گاهی به این شکل:
اما در نهایت شکل خط فرمان شما در کاربری آن تفاوت خاصی ایجاد نمی‌کند.
\begin{itemize}
	\item فرق دلار با هش
	\item راهبری عمومی و کلیدها
	\item مفهوم بش و سی اس اش و غیره
	\item پایپ کردن
	\item ته دستور باید انتر زد!
	\item از خیلی از دستورات می شه با 
\lr{ctrl+c}
 بیرون اومد یا زدن
\lr{ctrl+d}
  به شیوه ای معقول تر
	\item کامل کردن با تب و اهمیت اون
	\item تفاوت داشتن حروف کوچیک و بزرگ
\end{itemize}
\begin{mybox}
اگر در محیط
\lr{KDE}
هستید می‌توانید از 
\lr{Yakuake}
و اگر در محیط گنوم/یونیتی هستید از 
\lr{Guake}
برای داشتن یک خط فرمان زیبا که با زدن یک کلید (معمولا 
\lr{F12}
) از بالا ظاهر می‌شود، استفاده کنید.
\end{mybox}