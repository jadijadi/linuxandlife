\section{دیتاسنترها}
من اولین کار جدی‌ام رو که قبول کردم، یک دلیل داشت: ممکن بود بعضی روزها توی دیتا سنتر کار کنم. فکر اینکه خواهم تونست کنار سرورهای سان، آی بی ام و اچ پی باشم و در کنار لینوکس‌های خودم با سیستم‌عامل‌هایی مثل سولاریس، اچ پی یو ایکس و غیره کار کنم به نظرم فوق العاده بود. دفعات اول فضا برام خیلی عجیب بود. کیسه‌هایی که دور کفش می‌کشیدیم، رک‌های بزرگ، کی وی ام ها و صدای دائمی \lr{woooovvvvv} کامپیوترها که بعد از بسته شدن در دولایه تنها چیزی بود که توی گوش می‌چرخید.

اما همیشه فضا اینطوری نموند. راه دور، سرما، ممنوع بودن خوردن و آشامیدن و آنتن ندادن موبایل‌ها خیلی زود شروع کردن ناراحت کننده شدن و بعد از یکسال منم مثل بقیه تیم نه فقط مشتاق رفتن به بالا سر سرورها نبودم که همه تلاش و اصرارم رو داشتم که باید با وی.پی.ان. داشته باشم و از راه دور بتونم به سرورها وصل بشم. وقتی هم به کشورهای دیگه سفر کردم دیدم در بسیاری از کشورهایی که سخت‌گیری کمتری روی کیفیت اجرای پروژه‌ هست، تیم‌های اجرایی با گذاشتن یک دونه لپ تاپ در رک مورد نظر و اجرا کردن برنامه‌هایی مثل تیم‌ویور اصولا بدون سر و کله زدن با مشتری، از راه دور به سیستم‌هاشون وصل می‌شن.

اینکار مشکل امنیتی داره! هیچ وقت انجامش ندین مگر اینکه مشتری تاییدش کرده باشه.

نتیجه‌اینه که کار کردن توی دیتاسنتر ترکیبی است نکته‌های مثبت و منفی. در حوزه منفی، از سرما و صدا و گشتن به دنبال صندلی برای نشستن و کابل برای وصل شدن به سرور. صدای فن کامپیوتر خودتون رو صد برابر بیشتر کنین و کولر روی ۱۲ درجه تنظیم کنین و جلوش بشینین تا بتونین درکی از شرایط دیتاسنتر داشته باشین. در مقابل حوزه مثبت اینجاست که خودتون تصمیم می‌گیرین چیکار کنین، مدیرها بالا سرتون نیستن، معمولا تنها هستین یا با تیم خوبی کار می کنین و به کامپیوترهای جالبی دسترسی دارین و تجربه‌ای رو می‌کنین که عکس‌هاش و خاطره‌هاش آب از دهن هر تازه‌واردی سرازیر می‌کنه.

ولی برای چنین تجربه‌ای نیاز به چه مهارت‌هایی دارین؟ شاید براتون جالب باشه که مهارت‌های غیرفنی حتی مهمتر از مهارت‌های فنی هستن. بذارین مرور کنیم.
\subsection*{قدرت شخصیت و اعتماد به نفس}
توی دیتاسنتر شما تنها هستین و خیلی وقت‌ها لازمه بتونین شخصا تصمیم بگیرین و اجرا کنین. حتی خیلی وقت‌ها ممکنه داخل دیتاسنتر به اینترنت هم دسترسی نداشته باشین (واقعا!‌ موبایل هم گاهی توی کانکس‌های بزرگ دیتاسنترها یا زیر زمین آنتن نمی‌دن) و در نتیجه باید داکیومنت و سواد و غیره رو بریزین روی لپ‌تاپ تا کاملا خودکفا بشین. راستش رو بگم من حتی در جاهایی کار کردم که اجازه نداشتین لپ‌تاپ و موبایل هم توی دیتا سنتر ببرین (:

همزمان بحث‌های روانی هم مطرحه. یک نفر ممکنه به راحتی در یک محیط پر از نویز سفید و باد سرد و دیوارهای روشن دچار توهم بشه یا نتونه به شکل درستی تمرکز کنه. چنین آدمی مطمئنا به درد کار توی دیتاسنتر نمی‌خوره.
\subsection*{مهارت‌های فنی}
منطقا اگر قراره کسی به شما حقوق بده که برین توی یک دیتاسنتر، قراره یک کاری اونجا بکنین. ممکنه متخصص شبکه‌ باشین، ممکنه تو کار کابل‌کشی باشین (که اتفاقا بسیار جالبه و بانمکه) یا ممکنه سیستم‌عامل بلد باشین. حوزه‌های خیلی خیلی زیادن. شاید شما متخصص یک نرم افزار خاص باشین که قراره نصب یا ساپورت بشه، ممکنه سخت افزار بلد باشین و ممکنه اطلاعات زیادی در مورد سخت‌افزارهایی مثل سرورها، استوریج‌ها (هاردها) و این تیپ چیزها داشته باشین.

اگر بحث مدرک است مدارک شبکه و چیزهایی مثل \lr{LPIC‌} می‌تونن شما رو وارد دیتاسنترها کنن و البته دوره‌ها و مدارک طراحی دیتاسنتر چیزی هستن که کلا بقیه زندگی شما رو وابسته به دیتاسنتر خواهند کرد.

اگر از من می‌شنوین سراغ شغل‌هایی برین که باعث می‌شه گاهی توی دیتاسنترها باشین و گاهی بیرون تا حق انتخابتون حفظ بشه. خوندن دیتاسنتر دیزان چیزی شبیه به خوندن کاپیتانی کشتی است؛ توصیه می‌شه قبل از خوندن حداقل یک بار سوار کشتی شده باشین (:

\subsection*{قدرت فیزیکی}
کسی که در دیتا سنتر کار می کنه معمولا لازمه صندلی‌اش رو خودش جابجا کنه، کمی کابل بکشه، در رک رو باز کنه - که گاهی سنگینه و حتی لازم می‌شه سروری رو از رک بیرون بکشه یا کی.وی.ام (دستگاه اتصال به سرورها) رو بیرون بیاره و درش رو باز کنه. شاید کلیت کار گاهی تفاوتی با کار در یک خط تولید نداشته باشه. بحث‌های سرما و خشکی هوا هم هست و منطقا اگر مدتی توی دیتاسنتر باشین، احساس گلودرد اصلا غیرعادی نخواهد بود. از اونطرف دیتاسنترها معمولا توی جاهای دورتر از مرکز شهر هستن و باید حوصله رفت و آمد رو هم داشته باشین.
\subsection*{اطلاعات اولیه از دیتاسنتر}
خب این رو اکثرا تا وقتی وارد دیتاسنتر نمی‌شیم نداریم اما دوره‌هایی هستن که این چیزها رو آموزش می‌دن. شاید توی همین مقاله هم بعضی عبارت‌ها برای شما کمی عجیب بودن ولی در مراجعه دوم و سوم به دیتاسنتر، خیلی از اونها رو یاد خواهید گرفت. چنین دوره‌های رایگان آنلاینی \LTRfootnote{\href{http://www2.schneider-electric.comsites/corporate/en/products-services/training/energy-university/energy-university.page) }{www2.schneider-electric.com}} سعی می‌کنن بخشی از این چیزها رو به شما آموزش بدن.

تجربه‌ای دارین یا لازمه چیزی اضافه بشه؟ به \lr{jadijadi} روی جیمیل خبر بدین (:
